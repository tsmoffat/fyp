Synthetic aperture radar (SAR) is a method of radar imaging that allows for a greater dimension of radar aperture than is possible in a single element or an array of elements. This is generally achieved by mounting a radar transceiver to a moving platform and moving it past the target to be imaged however the transceiver can be stationary and the target moved past it. The image uses the relative movement between the platform and the target to build up an image based on the responses received and then time multiplexing all the responses. This movement, and hence the larger aperture created, allows for a finer spatial resolution than can be achieved by more conventional beam scanning radar. These images can be useful for a variety of applications such as remote mapping of the surface of the Earth for glaciology and geology among others. Interferometric SAR can also be used to create elevation maps by taking two measurements from different positions and comparing the differences.
Simulating a SAR image is desirable as flight time can be expensive and time consuming to get quality images so defining the flight paths and angles before any imaging is carried out saves both time and money.
\par
There are three main methods that have previously been explored for simulating SAR. The first of these, which is the most accurate is simulating the effects of the beam itself, as detailed in Franceschetti \textit{et al.} 2003 \cite{franceschettiSARRawSignal2003}. This describes creating the raw data as would be expected in a SAR system in the real world and then transforming it to create a SAR image. The ground work for this was laid in Franceschetti \textit{et al.} 1992 \cite{franceschettiSARASSyntheticAperture1992}, which itself builds on Francescetti and Schirinzi's work on a SAR processor based on two-dimensional FFT codes \cite{franceschettiSARProcessorBased1990}. This approach has the advantages of being able to take into account backscattering as well as the higher-order signal contributions created by scattering as the signal reflects off of both a wall and the ground. A lot of the preliminary work with regards to the behaviour of electromagnetic backscattering was conducted in \cite{franceschettiCanonicalProblemElectromagnetic2002}. This describes a model that allows for the return from a structure to a microwave sensor to be analysed and so determine its dielectric properties as well as its geometric properties. The optics can be altered to simulate the roughness of the surface the signal is reflected off of last before it is received by the sensor. This can also be expanded to simulate backscattering. This simulator as a whole works very well with individual objects, however the simulation time was found to scale linearly with the number of objects present in the scene. Simulations in \cite{franceschettiSARRawSignal2003} were carried out using a Pentium IV processor from Intel Corp, released in 2001 which means that the simulation times achieved are not necessarily reflective of the performance achievable on more modern processors, however at the time it was found that while one object in a 512x512 pixel image required approximately 34 seconds, two objects in the same sized image required 1'02" and 16 objects increased this computation time up to 7'38". This is not ideal if the aim is real-time or even near real-time simulation capabilities as most urban environments that are to be simulated will contain far more than 16 structures. This efficiency does seem to have been improved in Franceschetti \textit{et al.} \cite{franceschettiSimulationToolsInterpretation2007} as in this paper it is used to simulate a SAR image of a $400\times 600 \textrm{m}^2$ area of the centre of Munich, incorporating the Technische Universität and the Alta Pinakothek, which leads to a reasonably complicated scene.  \par 
A similar approach building on the previously mentioned work is presented by Zheng \textit{et al.} \cite{zhengSimulationMethodSAR2008} which uses the scattering model in order to accurately simulate a SAR scene. This is achieved by making the assumption that the scene is made up of vertical buildings distributed on a rough dielectric terrain. This also discusses the computation of scattering coefficients under different conditions. As with the previous approach (and the following raw SAR approach) the Kirchhoff approach is used in this case.
\par
A similar approach to robust SAR simulation is described in \cite{delliereSARMeasurementSimulation2007}, which uses an electromagnetic approach closely following Maxwell's equations to create a finite-difference time domain method of simulation, which differs from \cite{franceschettiSARRawSignal2003} as that instead manipulates the signal in the frequency domain in order to create the raw SAR signal. This system has similar drawbacks to Franceschetti's approach with regards to computation complexity and time required, but is even more computationally complex, due to its ability to handle dispersive materials as well as phase changes. This approach could be of some interest for creating an incredibly robust simulator due to the advances in computing performance between the publication date and today. \par
There are two methods that are less complex computationally for simulating the image achieved by SAR. These both arise from graphics manipulation and more specifically 3 dimensional modelling. These are rasterisation and ray tracing. Rasterisation is the process of taking a 3 dimensional area and creating a 2 dimensional image using some sort of image transformation based on the location of the camera to the object in the 3D area. Ray tracing follows a similar process to light, just in reverse. It sends beams out from the camera in all directions that the camera can see and simulates them reflecting off of objects until they hit a source of illumination. Using this the rendering engine can determine if an object is illuminated or not and adjust appropriately. These two approaches have both been used in the past for simulation of SAR images. Rasterisation was used by Balz 2006 \cite{balzRealtimeSARSimulation2006} and ray tracing was explored by Auer \textit{et al.} 2008 \cite{auerRayTracingSimulating2008} and Mametsa \textit{et al.} 2002 \cite{mametsaImagingRadarSimulation2002}. This second paper was used in Hammer \textit{et al.} \cite{hammerComparisonSARSimulation2008} along with Balz 2006 to compare the relative merits of these two approaches. Ultimately the advantages of Balz's approach as detailed in \cite{balzImprovedRealTimeSAR2006} and \cite{balzRealtimeSARSimulation2006} is that the simulator is real-time so can be used to determine optimal azimuthal directions when recording a SAR image using a physical airborne platform. This form of simulation however doesn't take into account the contributions of higher order reflections so if it is being used for a demonstration tool, the images produced are less representative of an actual SAR image. This also means that corners aren't visible in the image, meaning that this form of simulation can't be used to test feature extraction algorithms. The two ray tracing simulators tested within \cite{hammerComparisonSARSimulation2008} can simulate these higher-order contributions meaning they can be used to test feature extraction algorithms, however due to the computationally intensive nature of ray tracing these can't be simulated in real time, so this form of simulation is less useful for planning purposes. \par
All of the forms of simulation presented here so far are only as good as the models used, as generally the modelled buildings are assumed to be made of one material with a fixed dielectric constant, while the modelled ground is made of a different material. In \cite{hammerComparisonSARSimulation2008}, the buildings modelled are created entirely of stone, with a grass ground. In real life however this becomes more complicated as buildings do tend to be made up of multiple materials, and often in urban areas the ground is made up of a material with a similar dielectric constant to the buildings. This also doesn't include any forms of dispersive materials such as metal. It is unclear whether any of the simulators covered in \cite{hammerComparisonSARSimulation2008} can simulate these sorts of materials accurately, which is something that should be taken to account in the future. \par
Yet another method similar to the image processing approaches is presented by Lu \textit{et al.} \cite{luGPUBasedRealtime2009} and uses methods most commonly used by video games. In this case the raw mathematic SAR simulation is not carried out, similar to \cite{balzHybridGPUBasedSingle2009} and \cite{auerRayTracingSimulating2008} and uses an orthogonal projection to cast the 3D model to a 2D image based on the slant range of the camera. This again has the potential of improving the performance of image generation in incredibly complicated scenes such as city centres.
\par 
Chen \textit{et al.} 2011 \cite{chenRadarImagingSimulation2011} proposes a hybrid method of simulating these images using analytical models. In this approach, the contributions provided by backscattering are shown using an electromagnetic analytical model ultimately based on that described by \cite{franceschettiSARRawSignal2003}, and the object position vectors are given by a geometric model using ray tracing. In this paper the analysis is mostly focused on cylindrical or cylinder-like buildings, with some analysis devoted to flat-roof buildings as well. This approach has a potential to be more accurate than the ray tracing models proposed previously due to the electromagnetic models involved, but also be less computationally complex than the more canonical SAR image simulators. 
\par
Xu and Jin 2006 \cite{xuImagingSimulationPolarimetric2006} is a paper that is not immediately relevant to this situation due to it modelling natural environments for SAR simulation however this is potentially interesting and relevant due to the variety of materials present in urban environments to be modelled. This is an algorithm that can deal with randomly and heterogeneously distributed objects within an image with varying dielectric constants and can deal with them in a reasonable length of time. This is something that is worth exploring if there is time in the future as it would add robustness to the system as a whole. 






