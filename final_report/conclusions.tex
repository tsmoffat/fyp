\section{Final Remarks and Outcome}
%TODO: Add in mini project review - what went well, took less time, what would do differently if did it again
\subsection{Project Review}
\section{Further Work}
There is a significant amount of work that can be done on this model to improve it. %TODO
\subsection{Arbitrary Radar Cross Sections}
 The most interesting improvement that could be made to the system would be the creation of an RCS solver for the target model that can take into account the materials used when 3D modelling the target structure. This would allow for a much more realistic target output and would allow for arbitrary targets to be used. To begin with this could use a single material similar to the built-in MATLAB version but then could be expanded to take into account multiple materials and how they affect radar cross sections. This would require the use of a more advanced 3D model file format than the \gls{stl} file used by the MATLAB solver as STL files only relate the vertices of the modelled object and not the textures or materials applied to them in the modelling process. A file format like the 3DS format created by Autodesk would appear to work well for this purpose as this format is able to preserve textures. This could then use data acquired (either through modelling or experimental acquisition) on the electromagnetic reflectivity of a variety of materials to map a more exact RCS than has previously been possible. This is one possible alternative as using a pre-calculated RCS of the building may be limiting. Depending on the efficiency, the solver could be able to re-calculate the RCS in a short enough time that the demonstration model could still be run in a reasonable (within five minutes or so for a four second simulation) amount of time in order to be usable in a lecture or self-learning environment.
 
 \subsection{A More Robust Approach to Simulation}
 Really the biggest potential improvements to this model come from re-implementing every component from scratch. This would likely require a lot of work however the ability to be able to run an essentially bespoke SAR simulator would, in my opinion, probably be worth the effort. This would mean that the limitations placed on this project by using the pre-existing MATLAB blocks would be removed as the blocks could be reconfigured to run however they were needed to. This would require re-examining of every component and reimagining its function in a way that is hopefully easier to follow. The Unreal Engine is a video game engine released by Epic Games that is free to download and has been used by hundreds of video games, including Fortnite Battle Royale, since it was released in its first version in 1998. MATLAB already has a link into Unreal Engine 4 in the Vehicle Dynamics Toolbox and it should be possible to reverse engineer this link for use in other applications or otherwise Unreal Engine uses C++ as its programming language. \par What this means is that Unreal Engine would provide a useful platform to build a custom system that could work in a way much closer to that described in Franceschetti \textit{et al.} 2003 \cite{franceschettiSARRawSignal2003}, with radar pulses transmitted from an aeroplane across free space, to a target with the radar signature calculated in near-real time and then returned to the airborne platform. In this, every aspect of the system would be clear and known to the person implementing it and can then be related to the end user. It is still unclear whether MATLAB calculates higher orders of reflection than the first order and due to the failure of the RCS solver I was unable to test this. Using a system that was created bespoke these unknowns can be controlled for and explicitly implemented or excluded. The other advantage of this system would be that multiple approaches discussed at the beginning of the project can be directly compared against each other which has been done before in Hammer \textit{et al.} 2008 \cite{hammerComparisonSARSimulation2008} however to my knowledge it has never been done when all the source data is in the same rendering engine and also hasn't included raw SAR simulation before. This could be a very interesting form of comparison to see just how accurate the three forms of simulation are when compared to a reference SAR image and an accurately modelled 3D environment of the area the reference SAR image was captured over. \par 
 Using the Unreal Engine or another video game engine would also mean that the quality of visual aids would be improved. Both on a graphical fidelity standpoint but also being able to see more of what the aircraft is doing at any one time. A pre-rendered demonstration could be produced in order to show individual pulses and then the section that reflects what's actually going on within the simulation model could be improved massively as well just due to more powerful tools. The downside to this is there would be more 3D modelling involved with the creation of this set up however there are many shape libraries already existing that are compatible with video game engines, including a first party store provided by Epic Games for the Unreal Engine. This would reduce the requirement to create new 3D models from scratch as pre-existing assets can always be altered for those purposes. \par These improvements can also take the form of an improved background and terrain as well. The current system can only really every have a flat background as that's all the system can take into account. Yes it has coordinates in three dimensions however this doesn't make a lot of difference in the overall simulation, whereas with structures at different elevations this system can also model layover, foreshortening and shadow. This can further aid the comprehension of what, without visual aids, can be somewhat difficult concepts to grasp. Along with this, being able to simulate the effects of reflections off of different materials and multiple reflections can cause natural speckle in the resulting image as opposed to having to artificially induce speckle.
 
 \subsection{Developing Hybrid Methods}
 