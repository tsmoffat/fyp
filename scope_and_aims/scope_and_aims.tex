This project is to create a simulator that can show the transformation between a 3 dimensional model and a 2 dimensional image using synthetic aperture imaging radar mostly for use as a demonstration tool. This is a method of radar that generally uses the movement of the platform that the radar transceiver is mounted on to build up an image based on the range that the radar signal travels and the reflection characteristics of the surface it is reflected off of. This form of radar imaging has a higher resolution than a conventional radar transceiver as the aperture size is based on how far the platform travels during the time it is imaging a target, and so the aperture is far larger than could be physically constructed. 
\par
The main aim of this project is to create a very basic simulator that can take a cube as an input and output an accurate SAR image. This will require a few steps to achieve. The first will be to create a scene in the chosen programming platform, which as of writing this document is MATLAB and Simulink, created by MathWorks. The radar pulse signal will need to be simulated as well as effects of multipath reflection, and finally the response received by the simulated radar will need to be transformed to an image by use of backscattering or an Omega-K algorithm. A secondary aim is to simulate the inclusion of speckle within the image, which is a SAR image artefact created by the sum of contributions from a large number of scatterers, thereby increasing the realism of the output image.
\par 
More advanced aims of this project are to be able to simulate the more advanced scenes such as actual urban environments and to simulate the multipath reflection effects of the radar signal in this area. This will end up requiring a more efficient implementation so the final major aim is to implement either a ray tracing or rasterisation approach, which are common methods of transforming a 3 dimensional space to a 2 dimensional image, most commonly used in animation and video games. While in other SAR simulation implementations efficiency is something that is considered, this is not necessarily something that is within the scope of this project. Improvements in efficiency are desirable up to a point however accuracy and functionality are more important. This is especially true as some of the literature (as covered later) focuses on a real-time implementation which is desirable for some applications, especially in practical applications such as determining SAR platform flight paths. It is not so applicable here in a demonstration application.